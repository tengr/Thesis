%*******************************************************************************
%*********************************** First Introduction *****************************
%*******************************************************************************
%\chapter{Introduction}  %Title of the First Introduction
%\chapter*{Introduction}
\chapter{Introduction}  %Title of the First Introduction
%\addcontentsline{tableofcontents}{chapter}{introduction}

\ifpdf
    \graphicspath{{Introduction/Figs/Raster/}{Introduction/Figs/PDF/}{Introduction/Figs/}}
\else
    \graphicspath{{Introduction/Figs/Vector/}{Introduction/Figs/}}
\fi


%********************************** %First Section  **************************************
\section{Motivation}\label{section1.1} %Section - 1.1 
\subsection{Text Mining and Information Extraction}
Despite being unstructured, amorphous, and only human-understandable, text is still the primary media for exchange of information\cite{witten2005text}. The huge volume of existence of such unstructured textual data presents a huge challenge to computer-driven natural language understanding. Text mining is the process of getting information from natural language text using methods in computer science, linguistics, and statistics. Information extraction, in particular, deals with this very process of transforming the unstructured information to a queryable and structured format. In recently years, the biomedical science field has seen huge growth in the volume of literature publications. Despite being in textual format, within those publications there exist valuable empirical findings about human genomics and patients and diseases which should be used to add to human knowledge. \newline

Text mining is a technology that uses methods in computer science, linguistics, and statistics to analysis natural language text and get the information within.  text mining transforms the large amount of unstructured data into structured format so that we have more data to analyze. Text mining is used widely in areas like marketing, document verifying, among others.

\subsection{Biomedical Literature Growth and Biocuration}
MEDLINE as accumulated more than 22 million references to biomedical publications\cite{MEDLINE:2015:Online}.
 Biomedical science has been developing quickly over the years, and the amount of published literature has risen to a point that is beyond manually analysis. Since there might be contradicting theories existing in those literatures, 	limited amount of reading might be causing to draw the wrong conclusions. A systematic way of analyzing documents is really necessary. The relation extraction tool has a very promising applications for researchers and medical field workers, pharmaceutical companies, 	and the general public.
 The \emph{Biocuration Process} refers to the process in which people read biomedical articles and enter the entires to the database. 

\subsection{Relation Extraction} 
Relation extraction is a form of information extraction where the semantic relations between entities are extracted.
Specifically, in this project we focus on the relation extraction among other informations. Biomedical relations covers a wide range of knowledge in this field.

\subsubsection{Relation Extraction Procedures}
\subsubsection{Name entity recognition}

%********************************** %Second Section  *************************************
\section{Research Question}\label{section1.2}%Section - 1.2
The project looks to investigate the application of  knowledge extraction system ASM on relation extraction tasks, specifically regarding the relation extraction on the variome corpus.
In this project we focus on the relation extraction among other informations. Biomedical relations covers a wide range of knowledge in this field.


\section{Corpus construction}  

%********************************** % Third Section  *************************************
\section{Thesis Structure}\label{section1.3} %Section - 1.3
The thesis is organized as follows: chapter 1 will



