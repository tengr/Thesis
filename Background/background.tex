%*******************************************************************************
%*********************************** First Introduction *****************************
%*******************************************************************************
%\chapter{Introduction}  %Title of the First Introduction
%\chapter*{Introduction}
\chapter{Related Work}  %Title of the First Introduction
%\addcontentsline{tableofcontents}{chapter}{introduction}

\ifpdf
    \graphicspath{{Background/Figs/Raster/}{Background/Figs/PDF/}{Background/Figs/}}
\else
    \graphicspath{{Background/Figs/Vector/}{Background/Figs/}}
\fi

This chapter will explore the research related to this thesis. 

%********************************** %First Section  **************************************
\section{Relation Extraction} %Section - 2.1 
In general, relation extraction refers two the process of discovering a relationship between entities in text. In the domain of natural language processing, the relation can be semantic, syntactic, etc, with semantic relations being the most important for knowledge discovery. Relations can be uniary, binary and complext with complex relations sometimes intermingled with the concept of events, In this project we are mainly concerned with binary relations as shown

\section{Named Entity Recognition}
\emph{Named Entity Recognition}, or NER, is the task of identifying elements in text that belong to pre-defined categories like person, organization, etc. Specifically, NER in biomedical text mining aims at identifying thing like proteins, diseases, genes, etc. There is extra difficulty for NER in the biomeidical domain mainly because of the following reasons. 

quote: A survey of current work in biomedical text mining
This task has been challenging for several reasons. First, there does not exist a complete dictionary for most types of biological named entities, so simple text- matching algorithms do not suffice. In addition, the same word or phrase can
Recognising biological entities in text allows for further extraction of relationships and other information by identifying the key concepts of interest
refer to a different thing depending upon context (eg ferritin can be a biological substance or a laboratory test). Conversely, many biological entities have several names (eg PTEN and MMAC1 refer the same gene). Biological entities may also have multi-word names (eg carotid artery), so the problem is additionally complicated by the need to determine name boundaries and resolve overlap of candidate names.
Because of the potential utility and complexity of the problem, NER has attracted the interest of many researchers, and there is a tremendous amount of published research in this topic. With the large amount of genomic information being generated by biomedical researchers, it should not be surprising that in the genomics era, much of the work in biomedical NER has focused on recognising gene and protein names in free text.
quote: A survey of current work in biomedical text mining
%********************************** %Second Section  *************************************
\section{Pattern Based Methods}
Relation extraction started from the task of extracting protein protein interactions from text by pattern matching. 
First, a set of part-of-speech rules are applied to split the sentence into simple sentences, e.g. (P1 VB1 P2 VB2 CC P3) is splitted to P1 VB1 P2 and P1 VB2 P3. Next, a set of word patterns is applied to extract relations from these sentences. In addition, cite{Discovering patterns to extract protein–protein interactions
from the literature: Part II} proposed an idea of minimal description length, as in finding a a pattern set that has the most balance between high presion, short rule length/lower rule complexity by dynamic programming to optimize the rule set.  Pattern based methods has achieved quite respectable performances, but it has the limitations that it would have a tough time capturing the richness in expressions, such as the anaphora terms like pronouns 
\section{Co-occurrence based methods}%Section - 2.2
Many approaches though extract information about genes from scientific texts using only information about the co-occurrence of terms in a sentence or abstract (Chaussabel and Sher, 2002; Becker et al., 2003; Tanabe et al., 1999; Stapley and Benoit, 2000; Jenssen et al., 2001;Wren et al., 2004). Simply ways of indicating co-occurrence measure include point-wise mutual information, log likelihood ratio. More advanced measures include the associative concept space,  The main advantage of co-occurrence based methods is their simple implementation and low computational complexity.

cite{Co-occurrence based meta-analysis of scientific texts:
	retrieving biological relationships between genes}, 
\section{Rule-Based Methods}

\section{Kernel Based Methods} 
Kernels provide a similarity measure between two objects in some complex feature space. In contrast to feature based methods, kernel-based methods allow the original representation of the object to be retained and the kernel function will work out the similarity measure. For instance, a sentence maybe represented as a dependency graph, the feature based methods would want to select features such as number of nodes, edges, directions, etc, whereas kernel based methods allows feeding the two entire graph representations into a kernel function and output the distance measure.

%********************************** % Third Section  *************************************
\section{Feature Based Methods}  %Section - 1.3 
\label{section1.3}

\section{Semi-Supervised Methods}