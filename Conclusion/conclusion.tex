%*******************************************************************************
%*********************************** First Introduction *****************************
%*******************************************************************************
%\chapter{Introduction}  %Title of the First Introduction
%\chapter*{Introduction}
\chapter{Conclusion}  %Title of the First Introduction
%\addcontentsline{tableofcontents}{chapter}{introduction}

\ifpdf
    \graphicspath{{Conclusion/Figs/Raster/}{Conclusion/Figs/PDF/}{Conclusion/Figs/}}
\else
    \graphicspath{{Conclusion/Figs/Vector/}{Conclusion/Figs/}}
\fi

%********************************** %First Section  **************************************
\section{Conclusion} %Section - 1.1 	
General literature mining at the semantic level. A combination of many approaches.

\section{Future Work}
\subsection{Full System Adaptation}
Given the nature of the existing software system, I effectively wrote an adapter for the Variome Corpus, transforming the current data set to a format that the system is expecting. This is, of course, far from the best solution. However, system was developed solely for the purpose of event extraction tasks, and has numerous constraints and data format expectations as hard-coded strings and methods, to the extent that all of my efforts to change the code have failed. It is natural and understandable for academic software like this to be one-and-done for things like the shared task, yet I think it is in the interest of future system users and developers to have a well-documented, adequately extensible system in place, with room for tweaking algorithms and file formats with parameters. Ideally, the shortest path kernel will expose interfaces for users to adjust what is happening under the hood with a few parameters. For instance, the existence of trigger should be optional and can be set with a parameter applied to the both rule learning process and event extraction process. Even in the shared task 2013, the existence of triggers for events like relations or coherence is optional. Next, annotation format should be parametrized, the system should establish a relationship between user-set entity and relation format and the entities and relations. Moreover, the named entity recognition could be a dispatch-able unit of the system too with options of a user-given entities or the output of a named-entity recognizer. A perfect scenarios would be that the user can input a configuration file indicating annotation formats, entity types, entity output(the annotation file or from named entity recognizer), event/relations types and textual data. That way the system can be used for different kinds of tasks, possibly even beyond the domain of biomedical text mining and provide valuable feedback for the plausibility of the approximate subgraph matching paradigm.   
\subsection{Parameter Tuning}
With a fully flexible and controllable system in place, the users can tune the parameters like subgraph weights, thresholds, aggressiveness of optimization for the training set and test them on the test set. 
