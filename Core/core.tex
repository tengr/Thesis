%*******************************************************************************
%*********************************** First Introduction *****************************
%*******************************************************************************
%\chapter{Introduction}  %Title of the First Introduction
%\chapter*{Introduction}
\chapter{Related Work}  %Title of the First Introduction
%\addcontentsline{tableofcontents}{chapter}{introduction}

\ifpdf
    \graphicspath{{RelatedWork/Figs/Raster/}{RelatedWork/Figs/PDF/}{Introduction/Figs/}}
\else
    \graphicspath{{RelatedWork/Figs/Vector/}{RelatedWork/Figs/}}
\fi

%********************************** %First Section  **************************************
\section{Data Collection} %Section - 1.1 
In general, relation extraction refers two the process of discovering a relationship between entities in text. In the domain of natural language processing, the relation can be semantic, syntactic, etc, with semantic relations being the most important for knowledge discovery. Relations can be uniary, binary and complext with complex relations sometimes intermingled with the concept of events, In this project we are mainly concerned with binary relations as shown

\section{Algorithm}
Named entity recoginition is the process of 
Iis project 
cite
the performance of the subgraph matching method, as an instance-based learning strategy (Alpaydin, 2004), is dependent on having good training examples that express the events in a range of syntactic structures, cite

%********************************** %Second Section  *************************************
\section{Why do we use loren ipsum?} %Section - 1.2


%********************************** % Third Section  *************************************
\section{Where does it come from?}  %Section - 1.3 
\label{section1.3}
