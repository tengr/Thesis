%*******************************************************************************
%*********************************** First Introduction *****************************
%*******************************************************************************
%\chapter{Introduction}  %Title of the First Introduction
%\chapter*{Introduction}
\chapter{Core}  %Title of the First Introduction
%\addcontentsline{tableofcontents}{chapter}{introduction}

\ifpdf
    \graphicspath{{Core/Figs/Raster/}{Core/Figs/PDF/}{Core/Figs/}}
\else
    \graphicspath{{Core/Figs/Vector/}{Core/Figs/}}
\fi

%********************************** %First Section  **************************************
\section{Data Collection} %Section - 1.1 
Our dataset is the Variome Corpus\cite{verspoor2013annotating}, which is openly accessible. \footnote{\href{http://www.opennicta.com.au/home/health/variome}\url{http://www.opennicta.com.au/home/health/variome}} \citet{verspoor2013annotating} gave a detailed illustration of the document selection and annotation process. I will summarize the main points here.
\subsection{Background}
A major part of the current biomedical research lies in understanding the relations between human genetic variation and disease phenotypes. The \emph{Human Variome Project}, or \emph{HVP}, is a global initiative to collect all genetic variation information affecting human health\cite{ring2006human}. In particular, it acts as a liaison between individuals and organizations to integrate the genetic variants into databases that are open to the general public\cite{verspoor2013annotating}. The \emph{International Society for Gastrointestinal Hereditary Tumours (InSiGHT)}, is an international organization which aims to benefit patients with hereditary gastrointestinal(GI) tumours by research, education and personal assistance. In 2008, InSiGHT and HVP began a collaboration which propels InSiGHT to refine its process in the integration and interpretation of genetic variants. Consequently, a substantial effort was made to understand the mutation of mismatch repair(MMR) genes, the cause of Lynch Syndrome - one of the main syndromes of GI cancer\cite{silva2009mismatch}. A total of 10 full-text articles were selected from PubMed Central\textregistered  by searching the common Lynch syndrome genes. These documents are mostly about inherited colon cancer. The annotation schema, also known as the Variome Annotation Schema\cite{verspoor2013annotating}, include 11 entity types and 13 relation types. as can be seen in the table here
\begin{table}
	\caption{A nice looking table}
	\centering
	\label{table:nice_table}
	\begin{tabular}{l c c c c}
		\hline 
		\multirow{2}{*}{Dental measurement} & \multicolumn{2}{c}{Species I} & \multicolumn{2}{c}{Species II} \\ 
		\cline{2-5}
		& mean & SD  & mean & SD  \\ 
		\hline
		I1MD & 6.23 & 0.91 & 5.2  & 0.7  \\
		
		I1LL & 7.48 & 0.56 & 8.7  & 0.71 \\
		
		I2MD & 3.99 & 0.63 & 4.22 & 0.54 \\
		
		I2LL & 6.81 & 0.02 & 6.66 & 0.01 \\
		
		CMD & 13.47 & 0.09 & 10.55 & 0.05 \\
		
		CBL & 11.88 & 0.05 & 13.11 & 0.04\\ 
		\hline 
	\end{tabular}
\end{table}

In short, the corpus is Inspired by needs of inSIGHT database, but intended for broader applications. Documents relevant to the genetics of Lynch syndrome, which covers inherited colon cancer as well as certain other cancers. Selected with PubMed Central To train tools for mining genetic variation and its relationship to disease
Here in this information extraction task, we treat the manually annotated data as the \emph{gold data}, 
\section{Algorithm}
To separate Named Entity Recognition from Relation Extraction problem, the named entity annotations are provided in training, development and test sets. Effectively, this is just asking computer to extract possible relations between these entities without worrying how to recognize them accurately. Of course, named entity recognition is an important step in a real relation extraction task as cite cite, because biological named entity recognition is such a complex problem and sometimes the relation extraction success would depend on the accuracy of NER.
Named entity recoginition is the process of 
Iis project 
cite
the performance of the subgraph matching method, as an instance-based learning strategy (Alpaydin, 2004), is dependent on having good training examples that express the events in a range of syntactic structures, cite

\section{System Adaptation}


\section{Results}

